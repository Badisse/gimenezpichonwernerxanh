\documentclass{article}
\usepackage{amssymb}
\usepackage{mathtools}
\usepackage{graphicx}
\usepackage[english,francais]{babel}
\author{Charlie , Raph , Luc , Léo}
\title{Projet PLT}
\date{2020-10-02}

\begin{document}
\pagenumbering{gobble}
 \maketitle
 \newpage
 \pagenumbering{arabic}
\tableofcontents
\newpage
\listoffigures
\newpage
\section{Règles}
\begin{figure}[!ht]
\begin{center}
\includegraphics[scale=0.8]{fiugre1.PNG}
\end{center}
\caption{Exemple de jeu en 2D}
\label{Exemple de jeu en 2D}
\end{figure}
\subsection{Les phases de jeux}
\begin{itemize}
\item [--]Deplacement
\item [--]Combat
\item [--]Echange ( ou achat ) d'equipement
\end{itemize}
Début de partie : Phase de déplacement
Fin de partie : Phase de combat
\subsection{Phase de deplacement}
\begin{itemize}
\item [--]Entre chaque combat et échange
\item [--]Fin de la phase : rencontre avec un ennemi du groupe (ou d’un joueur), échange
\item [--]Le groupe (ou les joueurs) se déplacent sur une grille au tour par tour d’un nombre de case aléatoire
\item [--]Les ennemis apparaissent et se déplacent sur la grille
\item [--] La carte doit être explorée pour être révélée
\end{itemize}
\subsection{Phase de combat}
\begin{itemize}
\item [--]Commence à la rencontre d’un ennemi lors de la phase de déplacement
\item [--]Fin du combat : une fois que les joueurs ou ennemis n’ont plus de santé
\item [--]Chaque personnage choisit une action chacun leur tour
\item [--]Un combat gagné rapporte des points
\end{itemize}
\begin{figure}[!ht]
\begin{center}
\includegraphics[scale=0.8]{figure2.PNG}
\end{center}
\caption{Exemple de phase de combat}
\label{Exemple de phase de combat}
\end{figure}

\subsection{Phase d'échange}
\begin{itemize}
\item [--]Commence lorsque les personnages se trouvent dans une zone d’échange
\item [--]Fin de l’échange lorsque les joueurs quittent le menu, finit sur une phase de déplacement
\item [--]Un menu spécifique d’échange propose des items à acheter ou d’échanger de l’équipement 
\end{itemize}
\begin{figure}[!ht]
\begin{center}
\includegraphics[scale=0.8]{figure3.png}
\end{center}
\caption{Resume de toutes les phases de jeu}
\label{Resume de toutes les phases de jeu}
\end{figure}
\newpage
\section{Ressources}
Zones de cartes:
\begin{itemize} \item [--] Visibilité
	\begin{itemize} \item [--] Visibles
				\item [--] inexplorées
				\item [--] Explorées non visibles
	\end{itemize}
	\item[--]{Type de zone}
	\begin{itemize}
\item [--] Echange
\item [--] Combat
\item [--] Déplacement
\end{itemize}
\item [--]Décor de déplacement : 
\begin{itemize}
\item [--]Salle de TP
\item [--]Salle de TD
\item [--]Hall A,B,C,D
\item [--] Jardin
\item [--] Salle de partiel		
\end{itemize}
\end{itemize}	
Interface utilisateur avec la santé, la date (si on l’utilise), les caractéristiques
On peut faire des UML en ligne tous ensemble ?
Pour faire des UMLs propre on peut utiliser LucidChart
après on peut p-e faire des trucs a main(souris) levée sur whiteboard(awwapp/google draw) avant de faire du propre
les deux sont en ligne btw 
\end{document}